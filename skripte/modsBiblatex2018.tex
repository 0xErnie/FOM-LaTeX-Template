\usepackage{xpatch} 

\setlength\bibhang{1cm} 

%%% Weitere Optionen 
%\boolitem[false]{citexref} %Wenn incollection, inbook, inproceedings genutzt wird nicht den zugehörigen parent auch in Literaturverzeichnis aufnehmen

%Aufräumen die Felder werden laut Leitfaden nicht benötigt.
\AtEveryBibitem{%
\ifentrytype{book}{
    \clearfield{issn}%
    \clearfield{doi}%
    \clearfield{isbn}%
    \clearfield{url}
    \clearfield{eprint}
}{}
\ifentrytype{collection}{
  \clearfield{issn}%
  \clearfield{doi}%
  \clearfield{isbn}%
  \clearfield{url}
  \clearfield{eprint}
}{}
\ifentrytype{incollection}{
  \clearfield{issn}%
  \clearfield{doi}%
  \clearfield{isbn}%
  \clearfield{url}
  \clearfield{eprint}
}{}
\ifentrytype{article}{
  \clearfield{issn}%
  \clearfield{doi}%
  \clearfield{isbn}%
  \clearfield{url}
  \clearfield{eprint}
}{}
}

\renewcommand*{\finentrypunct}{}%Kein Punkt am ende des Literaturverzeichnisses

\renewcommand*{\newunitpunct}{\addcomma\space} 
\DeclareDelimFormat[bib,biblist]{nametitledelim}{\addcolon\space} 
\DeclareDelimFormat{titleyeardelim}{\newunitpunct} 
%Namen kursiv schreiben
\renewcommand*{\mkbibnamefamily}{\mkbibemph} 
\renewcommand*{\mkbibnamegiven}{\mkbibemph} 
\renewcommand*{\mkbibnamesuffix}{\mkbibemph} 
\renewcommand*{\mkbibnameprefix}{\mkbibemph} 
%Delimiter für mehrere und letzten Namen gleich setzen
\DeclareDelimFormat{multinamedelim}{\addsemicolon\addspace}%hinzugefügt. Ansonsten werden mehrere namen mit komma getrennt
\DeclareDelimAlias{finalnamedelim}{multinamedelim} 

\DeclareNameAlias{default}{family-given} 
\DeclareNameAlias{sortname}{default}  %Nach Namen sortieren


\DeclareFieldFormat{editortype}{\mkbibparens{#1}}
\DeclareDelimFormat{editortypedelim}{\addspace} 
\DeclareFieldFormat{translatortype}{\mkbibparens{#1}} 
\DeclareDelimFormat{translatortypedelim}{\addspace} 
\DeclareDelimFormat[bib,biblist]{innametitledelim}{\addcomma\space} 

\DeclareFieldFormat*{citetitle}{#1} 
\DeclareFieldFormat*{title}{#1} 
\DeclareFieldFormat*{booktitle}{#1} 
\DeclareFieldFormat*{journaltitle}{#1} 

\xpatchbibdriver{online} 
  {\usebibmacro{organization+location+date}\newunit\newblock} 
  {} 
  {}{} 

\DeclareFieldFormat[online]{date}{\mkbibparens{#1}} 
\DeclareFieldFormat{urltime}{#1\addspace Uhr}
\DeclareFieldFormat{urldate}{%urltime zu urldate hinzufügen
  [Zugriff\addcolon\addspace
  #1
  \printfield{urltime}]
}
\DeclareFieldFormat[online]{url}{\mkbibacro{URL}\addcolon\space <\url{#1}>}
\renewbibmacro*{url+urldate}{% 
  \usebibmacro{url}% 
  \ifentrytype{online} 
    {\setunit*{\addspace}% 
     \iffieldundef{year}
       {\printtext[date]{keine Datumsangabe}} 
       {\usebibmacro{date}}}% 
    {}% 
  \setunit*{\addspace}% 
  \usebibmacro{urldate}
  } 


\renewbibmacro*{date+extradate}{% 
  \printtext[parens]{% 
    \printfield{usera}% 
    \setunit{\printdelim{titleyeardelim}}% 
    \printlabeldateextra}} 

\DefineBibliographyStrings{german}{ 
  nodate    = {{}o.\adddot J\adddot}, 
  andothers = {et\addabbrvspace al\adddot} 
} 

\DeclareSourcemap{ 
  \maps[datatype=bibtex]{ 
    \map{ 
      \step[notfield=translator, final] 
      \step[notfield=editor, final] 
      \step[fieldset=author, fieldvalue={{{o\noexpand\adddot V\noexpand\adddot}}}] 
    } 
    \map{ 
      \pernottype{online} 
      \step[fieldset=location, fieldvalue={o\noexpand\adddot O\noexpand\adddot}] 
    } 
  } 
} 

\renewbibmacro*{cite}{% 
  \iffieldundef{shorthand} 
    {\ifthenelse{\ifnameundef{labelname}\OR\iffieldundef{labelyear}} 
       {\usebibmacro{cite:label}% 
        \setunit{\printdelim{nonametitledelim}}} 
       {\printnames{labelname}% 
        \setunit{\printdelim{nametitledelim}}}% 
     \printfield{usera}% 
     \setunit{\printdelim{titleyeardelim}}% 
     \usebibmacro{cite:labeldate+extradate}} 
    {\usebibmacro{cite:shorthand}}} 

    \renewcommand*{\jourvoldelim}{\addcomma\addspace}% Trennung zwischen journalname und Volume. Sonst Space; Laut Leitfaden richtig
    \hypersetup{hidelinks} %sonst sind Fußnoten grün. Dadurch werden Links allerdings nicht mehr farbig dargestellt

\renewbibmacro*{journal+issuetitle}{%
  \usebibmacro{journal}%
  \setunit*{\jourvoldelim}%
  \iffieldundef{series}
    {}
    {\setunit*{\jourserdelim}%
     \printfield{series}%
     \setunit{\servoldelim}}%
  \iffieldundef{volume}
    {}
    {\printfield{volume}}
  \iffieldundef{labelyear}
  {}
  {
  (\thefield{year}) %Ansonsten wird wenn kein Volume angegeben ist ein Komma vorangestellt
  }
  \setunit*{\addcomma\addspace Nr\adddot\addcolon\addspace}
  \printfield{number}
  \iffieldundef{eid}
  {}
  {\printfield{eid}}
}

\renewbibmacro*{postnote}{% 
  \setunit{\postnotedelim}% 
  \iffieldundef{postnote} 
    {\printtext{o.S\adddot}} 
    {\printfield{postnote}}} 
